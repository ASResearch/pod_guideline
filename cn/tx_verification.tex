% !TEX root = main.tex
\section{交易验证}
\subsection{基础}
基于第\ref{sec:leader_election}章,在已选举出提案者的情况下,交易验证对出块进行验证。首先给出拜占庭环境下交易验证的相关定义和特性。

\begin{definition}
(交易验证) 对于验证者集合$V$\footnote{类似于定义\ref{def:leader_election},此处$V$可为有限或无限集合。},包含$n$个节点,其中最多有$f$个拜占庭节点,即最少有$n-f$个诚实节点。所有节点必须从提案中最终做出决策,并且满足下述条件:
\begin{itemize}
	\item 一致性(Agreement):所有诚实节点的决策必定相同。
	\item 可终止性(Termination):所有诚实节点在有限的时间内结束决策过程。
	\item 有效性(Validity):选择出的决策值必须来自某个有效的提案。
\end{itemize}
\end{definition}

在同步网络环境下,任何交易验证协议,都需要满足上述条件\footnote{部分研究中将可终止性和有效性描述为活性(Liveness)。}。%而在异步网络环境中,通常

如\ref{subsec:intro_tx_verification}小节介绍,交易确认机制根据块是否具有终结性划分为链式协议和基于投票协议。投票式交易验证满足终结性,即交易一旦完成验证则不可能发生改变。对应地,链式交易验证则仅满足概率终结性,即随着链结构的增长交易发生改变逐渐降低,但无法为$0$。需要指出的是,即便是类似PBFT的投票协议,其数据结构仍可以采用链式区块结构。

从安全性角度,我们不希望已经验证的交易会存在修改的可能,即出现长距离攻击(Long-Range Attack)。同时,随着数据的不断增长,一些普通节点可能无法负担庞大的数据量,因此确定性的终结性可以减少数据的存储。最后,考虑到未来的分片设计,数据分片需要终结性作为基础。因此我们采用基于投票机制的设计,保证交易的终结性。

\subsection{投票效益}
对于投票协议,除了上述特性之外,我们认为其需要考虑如下因素:
\begin{itemize}
	\item 女巫攻击:投票过程必须能够抵抗女巫攻击(Sybil Attack),解决方法可以是通过设置准入门槛或者将投票效益与参与者的资产挂钩。
	\item 投票效益:通常而言区块链共识的投票是根据参与者的持有(或抵押)的资产计算所得,在没有投票成本的情况下,验证者更倾向分散其投票效益。一种解决方法是引入投票成本,或者对于分散投票行为作出惩罚\cite{buterin2017casper}。	
	%\item 共谋:为防止贿选以及共谋现象(Colluding),投票前投票节点身份不应曝光,同时在每次投票后(例如公布对某个区块的签名公证),其将丧失投票资格,直至再次入选委员会。
	%\item 延迟:尽管任何共识机制都需要保证在有限时间内结束决策,但我们仍希望投票能够快速达成一致,从而减少交易的确认延时(confirmation delay)。
\end{itemize}

\subsection{投票}
首先我们给出投票行为的定义:
\begin{definition}
(投票) 对于验证节点$i$,已知某提案区块$B=\langle Tx_B,h(B^\prec),t_B,\varphi_B \rangle$,且满足$V(B,t)=1$,对所有验证节点广播消息$\langle pk_i,sign_{sk_i}(t,h(B),\pi_i) \rangle$。其中$sign_{sk_i}(\cdot)$表示基于节点$i$私钥的签名,$t$表示用户$i$时间戳,$h(B)$表示提案区块$B$的哈希值,$\pi_i$表示验证节点$i$的投票效益证明。
\end{definition}

在此基础之上我们分析可能出现的恶意投票行为。对于拜占庭验证节点$i$,无法伪造或篡改其他人的消息\footnote{通常而言,我们认为现有的签名算法可以保证信息无法篡改或者伪造。},但仍可能出现如下恶意行为:

\begin{itemize}
	\item 恶意投票:恶意验证者不发布任何投票或者对其他验证者发布不同的投票,例如对部分验证者发布$\langle pk_i,sign_{sk_i}(t,h(B_1),\pi_i) \rangle$,而对另一部分验证者发布$\langle pk_i,sign_{sk_i}(t,h(B_2),\pi_i) \rangle$,$B_1 \neq B_2$。
	\item 割裂网络:在采用Gossip协议传输的网络,恶意节点可能在收到其他节点的投票后不向其他节点转发该信息,导致原有投票信息无法广播到所有节点。
	\item 共谋:恶意节点在投票前或者投票过程中得知其他验证者身份,从而贿赂其他验证者使其投票决策发生变化。
\end{itemize}

前两种行为会导致决策无法收敛,从而影响共识的活性。而最后一种行为虽然不会影响共识活性,但可能导致潜在的安全问题。




