\section{投票系统}
\label{ch:voting}
目前关于电子投票(Electronic Voting)的研究中\cite{kiayias2002self},除去传统的隐私性(Privacy)、公平性(Fairness)和健壮性(Robustness)需求,理想的电子投票还需要满足如下要求:
\begin{itemize}
	%\item 隐私性:
	%\item 健壮性:
	%\item 公平性:
	\item 可校验性(Universal-Verifiability):任何第三方都可以验证最后的投票结果是否正确统计了合法选票\footnote{另有原子可校验性描述仅投票者可以验证投票结果是否正确统计了合法选票。};
	\item 无收据性(Receipt-Freeness):投票者无法向第三方证明其所投的选票内容;
	\item 无争议性(Dispute-Freeness):任何第三方都可以验证协议的参与方是否正确执行了协议;
	\item 自计票性(Self-Tallying):任何第三方可以进行计票,而不需要可信第三方或者投票者的参与;
	\item 完善保密性(Perfect Ballot Secrecy):假设存在$n$个选民,任何$t$个($t<n$)投票者的投票结果只有剩余$n-t$个投票者串通起来才能知道。
\end{itemize}

而对于共识机制中的投票系统,由于不存在可信的第三方机构,每个验证节点既是投票者也是计票者\textcolor{red}{(即所谓的"all voters are talliers")},因此其必须满足可校验性和自计票性。\textcolor{red}{论文\cite{kiayias2002self}指出在大规模的投票系统中,并不需要满足完善保密性。}

现有大多数基于投票的共识机制都无法满足无收据性和无争议性,具体地来说,对于拜占庭验证节点,虽然无法伪造或篡改其他人的消息\footnote{通常而言,我们认为现有的签名算法可以保证信息无法篡改或者伪造。},但仍可能出现如下恶意行为:

\begin{itemize}
	\item 恶意投票:恶意验证者不发布任何投票或者对其他验证者发布不同的投票,例如对部分验证者发布$\langle pk_i,sign_{sk_i}(t,h(B_1),\pi_i) \rangle$,而对另一部分验证者发布$\langle pk_i,sign_{sk_i}(t,h(B_2),\pi_i) \rangle$,$B_1 \neq B_2$。
	\item 割裂网络:在采用Gossip协议传输的网络中,恶意节点可能在收到其他节点的投票后不向其他节点转发该信息,导致原有投票信息无法广播到所有节点。
	\item 共谋:恶意节点在投票前或者投票过程中得知其他验证者身份,从而贿赂其他验证者使其投票决策发生变化。
\end{itemize}

\textcolor{red}{作为分布式系统,区块链中所有验证节点通过P2P方式进行通信,因此任何节点在投票过程中都无法检验其他节点是否正确执行协议,即在投票过程中无争议性无法保证。}\footnote{尽管Casper通过Slash机制实现了检点的互相监督,但恶意行为被发现是基于交易已经上链的前提。}在这种情况下,恶意投票和割裂网络将变得可行。这两种行为会导致决策无法收敛,从而影响共识的活性。%目前的投票共识往往通过监督检举的方式来减少恶意投票和割裂网络行为\cite{buterin2017casper}。

共谋行为则违背了无收据性,\textcolor{red}{现有大部分投票共识都没有考虑拜占庭节点的共谋行为}并且假设系统中拜占庭节点比例低于某个阈值(例如三分之一),而实际上,由于共谋行为的存在,系统中拜占庭节点比例会更高。