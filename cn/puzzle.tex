\section{谜题选择}
\label{ch:puzzle}
谜题(puzzle)在PoW协议中扮演重要角色。通常,PoW协议规定只有解决给定谜题的矿工拥有出块权,是一种抵抗女巫攻击的有效手段。有文章指出PoW本质是通过谜题实现一个分布式时钟\footnote{https://grisha.org/blog/2018/01/23/explaining-proof-of-work/}。

谜题的选择同样也面临各式各样的取舍:
\begin{itemize}
	\item 谜题固然需要一定的难度来防止女巫攻击,但另一方面,有研究\cite{zeng2019matthew}表明对于任何算力竞赛模型,高难度的谜题存在马太效应,更容易造成大户垄断(51\%dominance)。
	
	\item 谜题的难度固然需要动态调整以适应不断升级的算力,但另一方面,动态难度会造成跳链现象(见章节\ref{ch:attack})。同时,动态难度也会遭受所谓长程攻击(long-range attack),即攻击者从某远古区块开始一直以极低算力挖一条私链,因难度是动态的私链增长速率可以和主链一致,然后攻击者一定时间段突然加大算力,使私链长度大于主链。通常解决长程攻击的方式为矿工检测到分叉时,除简单的采取最长链原则外同时也要检测区块难度,以摒弃难度过低的链。
\end{itemize}

鉴于谜题在PoW协议中的重要作用,作为区块链共识设计者,了解包括比特币在内的多个PoW所采用的谜题及工作原理,优势劣势等,也是设计合理的PoW共识必不可少的一部分。

本章节主要针对Mimblewimble共识协议(Grin项目)所采取的谜题进行介绍并展开思考。该谜题名称叫cuckoo,发表在2015年FC上\cite{tromp2015cuckoo}。

比特币的挖矿工具经历了CPU,GPU,FPGA,ASIC四个阶段。现今有比特大陆等矿机公司已经本质上实现了比特币的算力垄断。而cuckoo旨在提出一种新的谜题,使得挖矿工具的更新停留在GPU这一步——只有1060以上显卡才能进行挖矿。

谜题的本质是验证(verification)与探索(proof attempt)的不对称性。比特币所采取的SHA256由于哈希函数的难逆性无疑符合条件。cuckoo采取的是随机图找环算法,通过引入内存带宽限制构建谜题的难度。具体步骤如下:
\begin{itemize}
	\item 图的生成。
	
	二部图的$N$个点已经给定,通过哈希函数随机生成二部图的$M$条边(大约$N/2$)。生成边的方式要满足一定的要求,可以理解为每条边是根据$(k,nonce)$的SHA512值决定,其中$k$为编号,$nonce$为矿工尝试的数字。验证时,一旦给出nonce则可还原出图中所有的边。
	
	\item 谜题目标。
	
	给定一个图,矿工需要给出一个长为$L$的环。验证时,一旦给出图以及$L$个点的编号,可以轻易验证环是否能形成。
	
	值得一提的是,从一个图里面寻找长为$L$的环是多项式时间可解的\footnote{$L$为偶数时,时间复杂度为$n^2$。$L$为奇数时,时间复杂度为$M(n)$,其中$M(n)$为计算矩阵乘法所需的时间复杂度。}。然而由于图的生成是完全随机的,且每个图大概率不存在长为$L$的环,故矿工仍然需要暴力搜索。
	
	\item 找环推荐算法。
	
    文章推荐算法包含两部分。
    \begin{itemize}
    	\item 减支部分
    	
    	所谓减支本质上是完成一个拓扑排序问题:去掉所有度数为1的以及相邻的边,重复上述过程直到所有点的度数$\geq 2$。由于上述减支过程需要存储每个点的度数,故需要进行大量内存读取。这就是该谜题能引入内存带宽限制的原因。
    	
    	文章同时也提出了其他的减支算法,$BFS(L)$和$BFS(L/2)$,能避免对每个点都记录信息,但会消耗更多的时间,是一种时间和存储的平衡(TMTO,time-memory trade-off)
    	
    	\item 找环部分。
    	
    	文章推荐的找环算法维护一个有向图森林,以类似并查集的方式将边逐条加入。一开始,所有孤立点各自都是一座森林。一旦一条边加入,如果两个端点属于两个不同森林,则将两个森林合并,通过维护森林中边的指向与每个节点的root值。当且仅当新加入边的两个端点属于同一森林时,则必存在一个环,可根据有向图路径找到该环并确定长度。
    	
    	值得一提的是,如果找出来环的长度不为$L$,则忽略该条边继续上述操作。这样虽然可能导致有的长为$L$的环被漏掉,但这种情形概率不高,作为一种概率性的算法仍能保证高效性\footnote{高效概率性算法在实际运作中比比皆是。一个经典的例子是质数判定问题$Prime()$。有研究已经证明该问题是多项式时间可解,但时间复杂度仍然很高。实际运行时人们仍然选择用费马小定理进行判断。后者不能保证$100\%$正确但更快。另一个例子是线性规划问题,虽然已被证明椭圆算法能在多项式时间解决,但人们更多的还是采用单纯形法,后者不能保证多项式时间解决,但实际平均运行时间往往更低。}。
    \end{itemize}
\end{itemize}	

文章推荐的数据规模$N=2^{25}+2^{25}$,$L=42$

文章接下来给出了很多实验图表。这里不一一列出,仅简要介绍结论。
\begin{itemize}
	\item 计算哈希函数的时间开销随着节点规模增大而减小,最终低于$15\%$。(大部分时间用于找环)
	\item 存在$42$环的概率在$M/N>1/2$时剧烈增长。
	\item 内存的读开销随着已尝试nonce的百分比指数级上升,但写开销持平。
	\item 存在环的概率与$L$大约成反比。
\end{itemize}

总结:就目前而言cuckoo能限制矿机,但同时也需要考虑到新型矿机的可能性(如基于路由器的大带宽)。