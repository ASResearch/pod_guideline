\section{总结}
我们可以发现,区块链的共识和传统共识机制存在许多相似之处。例如对于区块一致性的达成本质上也是状态机复制,同时区块链共识在拜占庭环境下对安全性和活性的需求也并不能违背FLP定律。

但区块链相比传统分布式系统更为复杂,主要表现在以下几个方面。

\begin{itemize}
	\item 以比特币为代表的区块链系统摆脱了传统分布式系统的领袖-备份节点的设定,后者被认为是共识达成的瓶颈之一\cite{howard2019consensus}。在区块链系统中,出块节点负责发起对整个系统的状态更变,因此区块提议作为区块链共识机制中的重要部分被广泛研究,具体来说,理想的提议机制需要满足不可预测性以及合理出块频率。%尽管PoW由于能耗问题引起大量争议,但其仍被认为是目前最好的区块提议机制\footnote{https://zhuanlan.zhihu.com/p/52251671}。
	\item 区块链共识对于拜占庭容错的需求非常高,在大规模的异步网络下,应该优先保证安全性。同时,区块链共识需要保证已经达成一致的状态不会被改变(或者发生改变的概率非常低)。
\end{itemize}

与此同时,我们看到越来越多的研究关注于具有扩展能力(Scale-out)的区块链共识算法,在并没有严谨定义的Layer 1\&Layer 2概念中,分片技术(Sharding)被认为属于Layer 1机制,而闪电网络(Lightning Network)、状态通道(State Channel)等技术被认为属于Layer 2机制\footnote{https://lucidity.tech/layer-2-blockchain-technology/}。上述技术并不对区块链共识产生本质上的影响,我们将在后续工作中进一步阐述。

