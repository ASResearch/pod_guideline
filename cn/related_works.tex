% !TEX root = main.tex
\section{相关研究}
本章节简单介绍已有的区块链共识研究工作,这里并未按照一些公认的划分方法\cite{wang2019survey,bano2017consensus},而是分为区块提议和一致达成两部分介绍。同时受限于篇幅,本章并不会详细介绍各种研究,更详细的介绍可以参见星云共识调研报告\cite{nebulassurvey2019}。

\subsection{区块提议}
在传统分布式服务中,一般发起状态更变命令的节点称之为领袖节点(Leader)或者主节点(Primary),此类节点一般由发起请求客户端指定或者按照某种固定算法决定\cite{castro1999practical}。在区块链系统中,由出块节点(Block Proposer)负责提出区块,从而发起整个账簿的状态更变请求,但区块链中出块节点的选择和传统的领袖选举仍然存在较大的区别。%并且不同系统的选择机制也不尽相同。

%因此领袖选举实际上是对出块节点的选择\cite{gilad2017algorand}。根据区块链系统对于节点的设定不同,其采领袖选举机制也不尽相同。


%相比于传统分布式系统,在匿名和去中心化场景下区块链共识的领袖选举需要应对更多挑战。%因此多数已有的共识算法不包括领袖选举或者并未强调其重要性。

%\footnote{一些算法允许在领袖发生故障时进行领袖重新选举。}

%为保证匿名性,区块链中所有节点均表示为基于公钥产生的地址(例如以太坊中长度为64位的16进制字符串),
%同时为保证去中心化以及防止DDoS攻击,客户端产生的transaction会更新到全节点共享的交易池中(transactions pool)而无法指定特定的节点来发起提案(proposal)[]。


以比特币\cite{nakamoto2008bitcoin}和以太坊\cite{wood2014ethereum}为代表的区块链通常对于出块节点没有准入限制,参与出块的节点可以随时加入离开整个网络,此类区块链系统被定义为无授权区块链(Permissionless Chain)\cite{pass2017rethinking}。这类系统通过竞争方式选择出块节点,其中最为广泛采用的是工作量证明(Proof-of-Work,PoW)算法\cite{dwork1992pricing}。需要指出的是,PoW仅为出块节点选择协议,而并不是完整的共识协议\footnote{通常比特币采用的共识被称之为Nakamoto Consensus。},这一点在\cite{nakamoto2008bitcoin}中已经指出。

对于联盟链(Consortium Blockchain)以及部分采用DPoS机制的公链\cite{grigg2017eos},参与节点需要获得授权并且不能随意加入离开,因此被定义为授权区块链(Permissioned Chain)\cite{pass2017rethinking}。在此类系统中,一段时间内所有参与节点集合已知且不变,出块节点的选举可以采用轮询机制(Round-Robin),从而整个共识问题退化成传统分布式系统中的共识问题,即在出块节点的选择已经达成共识的情况下(此时出块节点仍可以采取恶意行动),所有验证节点如何针对出块节点的提案达成一致\cite{castro1999practical}。

%目前关于授权区块链是否能称之为真正意义上的区块链尚存在争议\footnote{EOS is not a blockchain, https://thenextweb.com/hardfork/2018/11/01/eos-blockchain-benchmark/},本文中的领袖选举特指无授权区块链的领袖选举。

%但可以肯定的一点是,无授权区块链更符合去中心化思想。不断有研究尝试解决无授权环境下的领袖选举问题。
由于PoW会消耗大量电力,权益证明(Proof-of-Stake,PoS)随后被提出,不同于PoW,后者实现出块选择往往与节点的链上资产相关而不依赖于物理算力因此都称之为PoS,但不同基于PoS的共识之间通常差别较大。

早期的PoS仍然是基于PoW的变种实现\cite{king2012ppcoin},即对谜题计算中的目标基于权益(Stake)或币龄(Coin-age)进行加权。作为备受瞩目的以太坊演进方案,Casper\cite{buterin2017casper}被认为是一种PoS协议,但实际上Casper是利用stake实现链式结构的最终性(Finality),并不涉及出块提议过程,因此基于Casper的共识仍然可以选择采用PoW机制。

%无授权区块链对于参与节点没有身份验证,因此对于领袖选取需要引入随机性,以保证安全\cite{pass2017rethinking}。

近几年涌现出一些基于随机数生成(Random Number Generation,RNG)的共识机制实现了基于PoS的出块提议方案\cite{gilad2017algorand,david2018ouroboros,hanke2018difinity}。尽管在出块方式和共识实现上存在区别,上述方案均利用伪随机函数(pseudo-random function)实现了出块节点的随机选则。

%同时根据[]的研究,在网络传输带宽有限的情况下,区块产生间隔和链分叉率存在相关性,因此在采用最长链等结构的区块链中,我们认为出块者选举机制也需要控制全网出块频率。

\subsection{一致达成}
\label{subsec:intro_tx_verification}
%在分布式系统中,状态机复制可以有效解决多服务器之间的数据一致性问题\cite{schneider1990implementing}。最为经典的Paxos协议实现了宕机容错(crash-fault tolerant)的一致性算法\cite{lamport2001paxos},其被广泛应用于数据库备份和域名服务等领域\cite{burrows2006chubby,chang2008bigtable}。
出块节点打包区块并广播后,其他节点对区块进行决策从而达成一致。

以比特币为代表的无授权区块链采取了最长链原则(Longest Chain Rule)\cite{nakamoto2008bitcoin},随后论文\cite{sompolinsky2015secure}提出了最重子树原则(GHOST Rule)。最长链原则和最重子树原则通常被称之为链式协议(Chain-based Protocols),在链式协议中,节点基于统一的规则构建链式账簿。

对于链式结构的系统,在达到最终一致性前任何区块状态都可能发生改变(例如某个区块在某个时间后不再属于最长链),通常这种改变概率随着链式结构增长单调递减\cite{nakamoto2008bitcoin},因此链式结构共识无法保证最终性(Finality),或者仅提供概率最终性(Probabilistic Finality)\cite{wang2019survey}。

%随着区块链和数字货币的发展,基于拜占庭容错的一致性算法重新得到关注。在大规模节点网络中,网络通信成为瓶颈。

以联盟链等项目为代表的非授权区块链通过缩小节点规模,实现了更强的一致性保证。节点针对每个提案进行投票,通常基于PBFT\cite{castro1999practical}或者某些更为宽松的投机BFT算法\cite{kotla2007zyzzyva},此类机制被称之为投票协议(Vote-based Protocols)或者BFT协议(BFT-based Protocols)\footnote{https://blog.cosmos.network/consensus-compare-casper-vs-tendermint-6df154ad56ae}。不同于链式区块链,采用投票协议的区块链会针对每个出块进行一致性确认,因此每个区块状态不可能被扭转,从而保证了最终性。

由于PBFT算法以及最差情况下投机BFT算法的网络通信开销为$O(n^2)$,因此采用此类协议的网络往往会控制验证节点规模\cite{castro1999practical,kotla2007zyzzyva}。近几年一些研究尝试在投票协议系统中扩展验证节点规模,例如ByzCoin\cite{kogias2016enhancing}和OmniLedger\cite{kokoris2018omniledger}使用了集体签名(Collective Signing)实现了大规模节点投票;Algorand\cite{gilad2017algorand}采用密码抽签(Cryptographic Sortition)方式可以从候选集合中抽取验证节点再投票。

较为特殊地,Casper\cite{buterin2017casper}的交易验证采用了一种链式协议和投票协议的混合策略。具体来说,Casper将某个特定区块高度的区块定义为检查点(checkpoints),在检查点之间的区块采用最长链原则,而对于每个检查点则通过投票来保证最终性。

%不同检查点间则通过投票保证终结性。

%基于PBFT机制,ByzCoin\cite{kogias2016enhancing}利用集体签名(Collective Signing)实现了投票节点规模可扩展,因而在一些分片技术(sharding)中也被采用。



%需要指出的是,通常认为采用链式协议的系统吞吐量(Throughput)低于基于投票协议的系统,实际上在仅满足最终一致性条件以及同等节点规模和网络环境下,链式协议实现状态同步开销更低。

%Bitcoin和Ethereum的成功表明最终一致性能够满足去中心化交易和图灵完备编程需求,







