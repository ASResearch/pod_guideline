% !TEX root = main.tex
\section{系统模型}
\subsection{网络模型}
异步网络设定 TBA

由于网络传输延时的存在,网络中的节点无法实现完美同步时钟(Perfectly Synchronized Clock)。因此TBC

在后文中,若没有特殊说明,$t$均表示节点的本地时间。

\subsection{数据模型}
在区块链中,最基本的数据结构是交易(Transaction)和账户(Account),这里给出相关定义。

\begin{definition}
(账户)一个账户$i$拥有一个私钥(Secret Key)$sk_i$和基于私钥构建的公钥(Public Key)$pk_i$。一个账户$i$可以表示为多元组$\langle pk_i,sk_i,s_i\rangle$,其中$s_i$表示用户$i$状态\footnote{在交易网络中,用户状态即用户余额,随着智能合约和用户行为的多样化,这里统一称之为用户状态}。
\end{definition}

以Bitcoin为代表的交易网络主要采用了未交易输出模型(Unspent Transaction Outputs Model,UTXO Model);而以Ethereum为代表的区块链系统采用了账户模型(Account Model)。通常来说,账户模型设计更接近于传统记账系统(例如银行),并且支持图灵完备智能合约(Turing-complete Smart Contract),因此这里我们基于账户模型给出相关定义,需要指出的是,UTXO模型理论上同样适用于我们的共识设计。

\begin{definition}
(系统状态)系统状态定义为系统中所有账户的状态,即$S=\{s_1,...,s_n\}$。
\end{definition}

\begin{definition}
(交易) 交易描述了一次系统的状态更变,一个交易包含交易发起账户$i$和接受账户$j$的状态更变,即$tx_k=\langle s_i,s_i',s_j,s_j'\rangle$,其中$s_i$,$s_j$表示交易发生前账户的状态,$s_i'$,$s_j’$表示交易发生后账户的状态\footnote{这里假设交易发生在一对账户间,对于多对多的交易可以转换为多笔交易}。交易也可能使账户状态不发生变化,即$s_i=s_i'$或者$s_j=s_j'$。
\end{definition}

在此基础之上,我们给出区块的定义。

\begin{definition}
\label{def:block}
(区块) 区块是一个数据结构,一个区块$B$包含了一组交易$Tx_B=\{tx_1,\dots,tx_n\}$,一个指向上一个区块的引用$h(B^{\prec})$,出块者的证明$\varphi_B$和\textcolor{gray}{时间戳$t_B$},其中$B^{\prec}$表示$B$所指向的前置区块,$h(\cdot)$为某种哈希函数。
\end{definition}


