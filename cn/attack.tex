\section{对PoW的进攻方式}
自比特币被提出同时成为最具影响力的区块链项目以来,对其核心的PoW协议的种种进攻方式的提出,以及相应的对策,相应安全性的分析从未停止过。目前大家熟知的就有“51\%攻击”(主要能实现“双花攻击”),以及2015年Eyal等人提出的“私自挖矿”攻击等等,同时还存在着鲜为人知甚至潜在的尚未被发掘的进攻方式。作为共识的设计者,深入了解此类进攻方式的工作原理及影响,并分析相应的对抗策略,无疑能为所设计共识的安全性提供极大帮助。

近期,Ren Zhang等人关于PoW进攻方式的研究的论文发表在安全领域顶级会议2019S\&P上(暂无引用链接)。这里我们以这篇论文为基础,简要概述其结论,并为本设计指导提出一些对PoW进攻方式见解与思考。

就目前而言最为广泛的进攻方式可大致分为两大类
\begin{itemize}
	\item 分叉攻击
	
	这类攻击方式就包括我们熟知的\textbf{双花攻击}:在攻击者$a$支付数字货币的交易$a\rightarrow b$被区块$B$打包,并且攻击者$A$获得实际(如线下收货)收益后,以$B$区块的父区块为根进行分叉产生新区块$B'$,同时$B'$包含攻击者将同一笔数字货币支付给其小号的交易$a\rightarrow a'$,进而达到否定区块$B$及其中交易$a\rightarrow b$的目的。结果$b$没能收到款但$a$收到了货。
	
	通常认为双花攻击需要攻击者达到全网$50\%$以上算力。
	
	同时也包括所谓的\textbf{私自挖矿攻击}。当攻击者挖出新区块后,并不选择立即公布这个区块,而是私自在新区块上继续挖矿并不断加长更新,即,维护自己的一条“私链”。当私链长度大于主链长度时,攻击者选择一直在私链挖矿。只有当私链长度等于主链长度时,攻击者才选择公布其私链并期望私链能赢得之后的算力竞争。结论表明只要攻击者算力大于全网的$1/3$,私自挖矿即可以让攻击者有利可图。具体分析见原论文\cite{eyal2018majority}。
	
	私自挖矿可以和双花攻击进行结合,即先在主链进行交易后再公布私链用以否定主链交易。
	
	\item 定点攻击
	
	这类攻击有个专业名称叫feather-fork,最早见于比特币论坛上\footnote{https://bitcointalk.org/index.php?topic=312668.0}。
	
	此类攻击允许攻击者在拥有即使小于$50\%$算力的情况下完全隔绝任何来自某特定地址(即所谓黑名单,如Alice)的交易。具体操作如下:
	
	攻击者事先做出一个承诺(commitment\footnote{commitment是博弈论中一个重要概念。见https://en.wikipedia.org/wiki/Stackelberg\_competition}):我永远不会在任何包含来自Alice的交易的区块上进行挖矿。攻击者会一直遵循他所做的承诺。
	
	作为一个普通矿工,当他听到攻击者的承诺后,作为一个利益最大化的个体,他在打包交易的时候也不会包含任何来自于Alice的交易:如果他的区块包含了Alice的交易,那么在攻击者拥有算力为$\alpha$(全网百分比)的情况下,至少有$\alpha^2$的概率攻击者连续挖到两个区块,这两个区块将接在该矿工区块的父区块上,使该矿工挖出的区块成为孤块而丧失奖励。而矿工不打包Alice的交易仅仅损失少量交易费。其结果是,所有理性矿工都会隔绝Alice的交易,达成所谓定点攻击。一般而言$\alpha$越大能拉拢的普通矿工越多。
	
	造成定点攻击的原因在于理性矿工与协议矿工(reference miner)的区别:理性矿工总会最大化自己的利益,而协议矿工会至始至终按协议运作。定点攻击只有在协议矿工的比例少于$50\%$时才作效。
	
	\item 其他攻击(跳链,矿池)。
	
	跳链严格来说不是一种攻击:因为比特币的挖矿难度是根据2周内平均挖矿时间动态调整的,那么,很多大矿工大矿池可以选择在比特币难度较高的时候转去其他PoW公链挖矿,待比特币难度降下来(必然结果。因为矿工跳槽了,总算力减少)之后再回归比特币。来源见于Bitcoin-NG~\cite{eyal2016bitcoin}
		
	所谓矿池相关攻击不属于共识层面,是因为矿池的引入导致各式矿工行为。这里稍微介绍下仅供参考。
	
	Pool-hopping~\cite{rosenfeld2011analysis}: 类似,有的矿池是根据单位时间收益(出块奖励/挖矿时间)来分配奖励。那么,矿工可以在某矿池挖了一段时间没挖出矿后跳到别的矿池去挖矿,因为在原矿池继续挖即使挖出来了因时间太长收益也低。一般为达到平均时间的$43.5\%$即跳槽。
	
	派间谍(Miners' dilemma~\cite{eyal2015miner}):简单而言,矿池$A$可以派一部分矿工,所谓间谍,去矿池$B$挖矿,但是间谍挖到真正的矿不会提交给$B$矿池,只提交挖矿的证明。(提交share,难度为矿的千分之一)。相当于,间谍从$B$矿池领工资但不真正挖矿,工资分给$A$矿池的人。(当然$B$矿池也会向$A$矿池派间谍,形成一个类似囚徒困境的局面)。
\end{itemize}

文章接下来分析了某些著名的PoW项目针对这些攻击的安全性,同时提出了几项评价指标。这里不详细介绍。其重点结论在于,针对上述进攻安全性不能同时满足:存在一个安全性悖论:“rewarding the bad and punish the good”。具体而言,对于区块链分叉,一般存在两种处理方式:

\begin{itemize}
	\item 对所有分叉同给予同样奖励,所谓“reward-all”。举例包括fruitchain,EthPoW(叔块)等。此类项目由于分叉没有损失,会加剧分叉攻击。
	
	\item 对分叉进行惩罚,所谓“punishment”,如将出块奖励均匀分发给各个分叉。举例包括DECOR+,Bahack‘s idea。此类项目由于分叉损失过大,理性矿工会更加担心自己挖出的块成为孤块,进而加剧定点攻击。
	
	\item 还有一类叫做“reward lucky”。此类协议奖励某些区块,定义比较模糊。举例如Subchains,Botail。但是文章认为lucy不等于good,也不能达到效果。
\end{itemize}

所以,该论文给我们的思考在于,设计共识应达成上述安全性的一个平衡。

文章最后给出的共识参考建议也值得一提:
\begin{itemize}
	\item 设计的协议不应该太复杂。
	\item 不应只针对特定的攻击来进行安全性分析。(应全面考虑)
	\item 不应针对特定攻击者奖励进行安全性分析。(应全面考虑)
\end{itemize}
另外,文章指出安全性能基于下面几项要素得到提高
\begin{itemize}
	\item 网络环境更好
	\item (弱)全局时钟的存在
	\item 可信赖的第三方
	\item 责任外包制度
	\item 基于“Layer 2”的抗攻击手段。
\end{itemize}

